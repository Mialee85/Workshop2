\documentclass[12pt, a4paper, onecolumn, oneside, toc=bibliographynumbered, liststotoc]{scrartcl} %Schriftgröße 12pt, DIN A4, 1 Spalte, einseitig bedruckt, Literaturverzeichnis in Inhaltsverzeichnis eintragen mit Nummerierung als Anhang

\usepackage[T1]{fontenc} %Codierung für deutsche Schriftzeichen
\usepackage[utf8]{inputenc} % UTF-8 Encoding
\usepackage[ngerman]{babel} % Neue Deutsche Rechtschreibung
\usepackage[onehalfspacing]{setspace} %1,5 Zeilen Zeilenabstand
\usepackage{scrlayer-scrpage} %Kontrolle von Fuß- und Kopfzeile
\usepackage{graphicx} %Einfügen von Bildern
\usepackage[printonlyused, nohyperlinks, smaller]{acronym} %Unterstützung für Abkürzungen und Abkürzungsverzeichnis - es werden nur verwendete Abk. gedruckt
\pagestyle{scrheadings} %Seitenstil
\chead*{\pagemark} %Kopfzeile Mitte - Seitenzahl
\cfoot*{} %Fußzeile Mitte - leer
\usepackage[german=quotes]{csquotes}
\usepackage[backend=bibtex, citestyle=authoryear, bibstyle=authoryear, sorting=nty]{biblatex} %Angaben für Zitate - Nutzt Bibtex, Markierung auf Seite als alphanumerische Abkürzung, Sortierung nach Auftreten
%\addbibresource{WissArb.bib} %Bibliotheksdatei
% !!! Um Bibtex richtig zu verwenden, nach jeder Änderung in der .bib-Datei Bibtex laufen lassen !!!

%\setcounter{tocdepth} {4} % Inhaltsverzeichnis bis subsubsection
%\setcounter{secnumdepth}{4} % Nummerierung des Inhaltsverzeichnis bis subsubsection

\begin{document}
%Titelblatt und Inhaltsverzeichnis
\pagenumbering{roman} %Seitennummerierung i, ii, iii etc. 
	%Angaben für Maketitle
	\titlehead{Hochschule Rhein-Waal \\ %Hochschulinformationen
	Fakultät: Kommunikation und Umwelt\\
	Studiengang: Verwaltungsinformatik\\
	Modul: XXXXXXXXXXXXXXXX\\}
%	\subject{Wissenschaftliches Arbeiten} %Art der Arbeit
	\title{TITEL} %Titel
%	\subtitle{Einfallstore für Black Hat Hacker in Netzwerke} %Untertitel
	\author{Linus Wolf}
	\date{\today} %Datum (heute)
%	\publishers{Betreut durch Professor Frank Zimmer} %Betreuender Professor und zusätzliche Infos

\maketitle %Erzeuge Titelblatt (Ignoriert in scrreprt voreingestellte Kopf- und Fußzeile)

\newpage %Ende Abstract
\tableofcontents %Erzeuge Inhaltsverzeichnis (Bei Fehler erneut kompilieren - ToC braucht 2 Durchläufe)

\newpage %Abschluss Titel und ToC - Neue Seite für Inhalt
\pagenumbering{arabic} %Seitennummerierung 1, 2, 3 etc. - Startet neu bei 1

	\section{Einleitung}

	\section{Dauerhafte Erreichbarkeit von Smart Home Geräten}
		\subsection{Mash-Netzwerk}
		\subsection{Zigbee}
%		\subsection{etc. bei Bedarf XXXXX}
	
	\section{Angriffsszenarien}
		\subsection{Hacking}
			\subsubsection{White Hat Hacking}
			\subsubsection{Black Hat Hacking}
		\subsection{Ausnutzung von Fehlern in der Sicherheit von Smart Home Geräten}
		\subsection{Man-in-the-Middle-Angriff}
		\subsection{Seitenkanalattacke}
%		\subsection{etc. bei Bedarf XXXXX}
		
	\section{Gründe für das Hacking}
		\subsection{Einfallstore in Netzwerke}
		\subsection{Botnetze}		
		\subsection{Ausspähen von menschlicher Anwesenheit}
	\section{Fazit}

\newpage %Abschluss Inhalt - Neue Seite für Anhänge
\appendix %Anhänge
\printbibliography %Erzeuge Literaturverzeichnis
\listoffigures %Abbildungsverzeichnis
\end{document}
